\documentclass[a4paper, 12pt]{article}
\usepackage[T2A]{fontenc}
\usepackage[utf8]{inputenc}
\usepackage[english,russian]{babel}
\usepackage{amsmath,amsfonts,amssymb,amsthm,mathtools}
\usepackage[papersize={192.6mm,230mm}]{geometry}
\geometry{top=2cm,bottom=3cm,left=4cm,right=3.5cm}

\usepackage{amsmath,amsfonts}
\usepackage[document]{ragged2e}
\usepackage{fancyhdr}
\usepackage{mathtools}
\justifying
\renewcommand{\headrulewidth}{0pt}
\renewcommand{\footrulewidth}{0.4pt}
\pagestyle{fancy}
\fancyfoot[C]{\thepage}
\setcounter{page}{267}

\begin{document}
\begin{justifying}

\noindent т.е. имеет место равенство(8.37). К этому можно лишь добавить, что из существования предела $\lim_{x\to x_{0}} \Psi(x)=1$ следует, что окрестность $U = U{(x_{0})}$ можно выбрать таким образом, что для всех точек $x \in X \cap U$ будет выполняться неравенство $\Psi(x) \ne 0$, а из существования по множеству X предела  $\lim_{x\to x_{0}}\frac{f(x)}{g(x)}$ следует, что окрестность U может быть выбрана еще так, что для всех $x \in X \cap U$ будет выполняться неравенство $g(x) \ne 0$, так как частное $\frac{f(x)}{g(x)}$ должно быть определено на пересечении $X \cap U$ множества X с некоторой окрестностью U точки $x_{0}$. Поэтому все написанные выше выражения имеют смысл.

Обе части равенства(8.37) равноправны, поэтому из доказанной теоремы следует, что предел, стоящий в левой части, существует тогда и только тогда, когда существует предел в правой части, причем в случае их существования они совпадают. Это делает очень удобным применение теоремы 2 на практике: ее можно использовать для вычисления пределов, не зная заранее, существует или нет рассматриваемый предел.\\

\noindent \textbf{8.4. Метод выделения главной части функции \newline и его применение к вычислению пределов}\\ 
\newline Пусть заданы функции $\alpha: X \to \mathbf{R}$ и $\beta: X \to \mathbf{R}$. Если функция $\beta$ для всех $x \in X$ представима в виде \[\beta(x) = \alpha(x) + o(\alpha(x)), x \to x_{0},\] то функция $\alpha$ называется \textsl{главной частью} функции $\beta$ при $x \to x_{0}$.

\textbf{Примеры. 1.} Главная часть функции $\sin x$ при $x \to 0$ равна x, ибо $\sin x = x + o(x)$ при $x \to 0$.

\textbf{2.} Если $P_{n}(x) = a_{n}x^{n} + ... + a_{1}x + a_{0}, a_{n} \ne 0,$ то функция $a_{n}x^{n}$ является главной частью многочлена $P_{n}(x)$ при $x \to \infty$, так как $P_{n}(x) = a_{n}x^{n} + o(x^{n})$ при $x \to \infty$.\newpage

Если задана функция $\beta$: $X \to \mathbf{R}$, то ее главная часть при $x \to x_{0}$ не определяется однозначно: согласно теореме 1, любая функция $\alpha$, эквивалентная $\beta$ при $x \to x_{0}$ является ее главной частью при $x \to x_{0}$

Например, пусть $\beta = x + x^{2} + x^{3}$. Так как, с одной стороны, $x^{2} + x^{3} = o(x)$ при $x \to 0$, то $\beta = x + o(x)$ при $x \to 0$, а с другой стороны, $x^{3} = o(x + x^{2})$ при $x \to 0$, поэтому \[\beta = x + x^{2} + o(x + x^{2}),  x \to 0.\] В первом случае главной частью можно считать $\alpha = x$, во втором $\alpha = x + x^{2}$. Однако если задаваться определенным видом главной части, то при его разумном выборе можно добиться того, что главная часть указанного вида будет определена однозначно.

В частности справедлива следующая лемма.\\
\textbf{ЛЕММА 5.}\textsl{Пусть $X \subset \mathbf{R}, x_{0} \in \mathbf{R}$ и $x_{0}$ $-$ предельная точка множества X. Если функция $\beta$: $X \to \mathbf{R}$ обладает при $x \to x_{0}$ главной частью вида $A(x - x_{0})^{k}, A \ne 0,$ где A и k $-$ постоянные, то среди всех главных частей такого вида она определяется единственным образом.}

Действительно, пусть при $x \to x_{0}$ \[\beta(x) = A(x - x_{0})^{k} + o((x - x_{0})^{k}), A \ne 0,\] и \[\beta(x) = A_{1}(x - x_{0})^{k_{1}} + o((x - x_{0})^{k_{1}}), A_{1} \ne 0.\]
Тогда $\beta(x) \sim A(x - x_{0})^{k}$ и $\beta(x) \sim A_{1}(x - x_{0})^{k_{1}}$ при $x \to x_{0},x\in X.$
Поэтому $A(x - x_{0})^{k} \sim A_{1}(x - x_{0})^{k_{1}}, x \to x_{0}, x \in X,$ т.е.\[ 1 = \lim_{x\to x_{0}}\frac{A(x - x_{0})^{k}}{A_{1}(x - x_{0})^{k_{1}}} = \frac{A}{A_{1}}\lim_{x \to x_{0}}(x - x_{0})^{k-k_{1}},\] что справедливо лишь в случае $A = A_{1}$ и $k = k_{1}$.

Понятие  главной  части функции полезно при изучении бесконечно малых и бесконечно больших и с успехом используется при решении разнообразных задач математического анализа. Довольно часто удается бесконечно малую\newpage 
\noindent сложного аналитического вида заменить в окрестности данной точки с точностью до бесконечно малых более высокого порядка более простой (в каком-то смысле) функцией.Например, если \(\beta(x)\) удается представить в виде \(\beta(x)=A(x-x_0)^k+o((x-x_0)^k)\), то это означает, что с точностью до бесконечно малых более высокого порядка, чем \((x-x_0)^k\) при \(x\rightarrow x_0\), бесконечно малая \(\beta(x)\) ведет себя в окрестности точки \(x\) как степенн\'aя функция \(A(x-x_0)^k\).

\justifyingПокажем на примерах, как метод выделения главной части бесконечно малых применяется к вычислению пределов функций. При этом будем широко использовать полученные соотношения эквивалентности (8.26).

Пусть требуется найти предел (а значит, и доказать, что он существует)
\[\lim_{x \rightarrow 0}
\frac{\ln (1+x+x^2)+\arcsin 3x-5x^3}{\sin 2x+\tg^2x+(e^x - 1)^5}.\]

Используя доказанную выше (см. соотношения (8.26)) эквивалентность \(\ln(1+u)\sim u\) при \(u\to 0\), имеем \(\ln(1+x+x^2)\sim \\ \sim x+x^2\) при \(x\to 0\), поэтому (см. теорему 1) \(\ln(1+x+x^2)=\\=x+x^2+o(x+x^2)\). Однако \(o(x+x^2)=o(x)\) (почему?) и \(x^2=o(x)\) при \(x\to 0\), следовательно,
\[ \ln(1+x+x^2)=x+o(x) \text{ при } x\to 0.\] 
Далее, \(\arcsin 3x\sim 3x\), поэтому
\[\arcsin 3x=3x+o(3x)=3x+o(x).\]

Очевидно также, что \(5x^3=o(x)\). Из асимптотического равенства \(\sin 2x\sim 2x\) получаем
\[\sin 2x=2x+o(2x)=2x+o(x),\]
из \(\tg^2x\sim x^2\) будем иметь
\[\tg^2x=x^2+o(x^2)=o(x),\]
а из \((e^x-1)^5\sim x^5\), аналогично,
\[(e^x-1)^5=x^5+o(x^5)=o(x).\]


\end{justifying}
\end{document}
